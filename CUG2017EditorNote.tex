\documentclass[AMA,LATO1COL,demo]{WileyNJD-v2}

\articletype{Article}%

\received{28 August 2017}
\revised{6 June 2016}
\accepted{6 June 2016}

\raggedbottom

\begin{document}

\title{Foreword to the Special Issue of the Cray User Group (CUG 2017)}

\author[1]{Scott Michael*}

\author[2]{Yun He}


\authormark{Michael \textsc{et al}}


\address[1]{\orgdiv{Pervasive Technology Institute}, \orgname{Indiana University}, \orgaddress{\state{Indiana}, \country{USA}}}

\address[2]{\orgdiv{Org Division}, \orgname{Org Name}, \orgaddress{\state{State name}, \country{Country name}}}

\corres{*Scott Michael, \email{scamicha@iu.edu}}

\presentaddress{2709 E Tenth St, Bloomington, IN, 47408, USA}

\abstract[Summary]{This special issue contains the ten most highly reviewed papers from
CUG 2017, the 60th meeting of the Cray User group. These papers address various topics
involved in supporting research, development, and operations on Cray systems. From
deploying and administering some of the world's largest scale systems, to designing and
optimizing applications and supporting users on these systems, the areas covered in the
five CUG technical tracks span a wide range. This year's CUG saw many emerging trends
being discussed and addressed including rising power utilization, increasing core counts
and scalability limitations, and the integration of new approaches that coupling data
analytics, HPC, and containerization in single environments. One new addition to the
technical program designed to give attendees opportunity to explore these new challenges
was a technical track focused on data analytics.
}

\keywords{}

\maketitle

\section{Themes of This Special Issue}\label{sec:themes}

From the accepted papers, and the conference in general, several themes emerged as being
at the forefront of deploying and supporting Cray systems. Although the conference is
roughly organized along five technical tracks (Filesystems \& I/O, Systems Support, User
Services, Applications \& Programming Environments, and Data Analytics) we saw some trends
that cut across these divisions as well as some highlights within these tracks.

\subsection{The Knight's Landing Architecture}\label{sec:knl}

The Cray XC40 is a system that supports Intel's latest many-core architecture codenamed
Knight's Landing (KNL). The CUG 2017 conference saw many presentations and papers
highlighting the opportunities and challenges in supporting and using moderate to large
scale XC40 deployments. Several of the papers in this special issue deal with topics
specific to KNL and the XC40. These include: \begin{itemize} \item Deployment and
acceptance of XC40 systems \item XC40 networking and KNL configuration options \item
Application performance on KNL \end{itemize} One of the biggest challenges that has been
faced by Cray users with respect to KNL is moving applications from one architecture to
another. Several of the papers in this issue address this with application comparisons and
some targeted best practices for moving to KNL.

\subsection{Optimization of I/O}\label{sec:io}

The best paper award at CUG 2017 went to a paper titled ``Lustre Lockahead: Early
Experience and Performance using Optimized Locking''. The focus of this paper on
optimizing performance of shared file writing on Lustre underscores the continued emphasis
on I/O optimization that we have seen in many past CUG conferences. Several papers in this
special issue focus on I/O optimization for large scale Cray systems.

\subsection{Data Analytics}\label{sec:data}

This year the CUG conference added a new Data Analytics technical track, reflecting the
emphasis throughout the HPC community on data analytics, machine learning, and big data
techniques. Although there is only a single paper from the Data Analytics track in this
special issue, there was a great deal of interest and discussion around Data Analytics at
the CUG conference.

\section{Conclusion}\label{sec:conclusion}

This year's Cray User Group conference saw many interesting developments and changes in
the computing landscape. Probably the biggest trend we saw for Cray machines was the
movement to KNL-based many-core architecture for large scale systems. The articles in this
special issue touch on many of the new developments in both the hardware and software
architecture. We hope that readers will benefit from the knowledge and insight into
operating large scale Cray machines contained in this issue.

\section{Guest Editors}\label{sec:editors}
\subsection{Scott Michael}\label{sec:scott}
Dr Scott Michael leads the Research Analytics team at Indiana University. His research interests include distributed processing frameworks, file systems and data storage technologies, data analytics, and machine learning. He is also a recovering astrophysicist. 
\subsection{Yun (Helen) He}\label{sec:helen}

\section{Acknowledgements}\label{sec:acknowledge}

We would like to thank all of the authors and all of the CUG participants who provided valuable contributions to this
special issue. We would also like to thank the members of the CUG Review Committee for the feedback provided to
the editors and authors, which was essential to selecting papers for publication and further improved many of the submissions. Finally, we would like to
thank Professors Geoffrey Fox and David Walker, the Editors, for
providing us with this opportunity to present our works in the international
journal of Concurrency and Computation: Practice and Experience.


%\nocite{*}% Show all bib entries - both cited and uncited; comment this line to view only cited bib entries;
\bibliography{CUG2017EditorNote}
\end{document}
